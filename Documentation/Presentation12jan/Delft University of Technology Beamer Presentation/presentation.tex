% TU Delft Beamer template
% Author: Maarten Abbink
% Delft University of Technology
% March 2014
% Version 2.0
% Based on original version 1.0 of Carl Schneider
\documentclass{beamer}
\usepackage[english]{babel}
\usepackage{calc}
\usepackage[absolute,overlay]{textpos}
\mode<presentation>{\usetheme{tud}}

\title[Beamer Sample]{Virtual Assistant}
%\subtitle
\institute[TU Delft]{Delft University of Technology Library}
\author{Soheil Jahanshahi}
\date{12 Januari 2015}

% Insert frame before each subsection (requires 2 latex runs)
\AtBeginSubsection[] {
	\begin{frame}<beamer>\frametitle{\titleSubsec}
		\tableofcontents[currentsection,currentsubsection]  % Generation of the Table of Contents
	\end{frame}
}

% Define the title of the "Table of Contents" frame
%\newcommand*\titleTOC{Overview}



\begin{document}

{
% remove the next line if you don't want a background image
\usebackgroundtemplate{\includegraphics[width=\paperwidth,height=\paperheight]{images/background-titlepage.jpg}}%
\setbeamertemplate{footline}{\usebeamertemplate*{minimal footline}}
\frame{\titlepage}
}

{\setbeamertemplate{footline}{\usebeamertemplate*{minimal footline}}
%\begin{frame}\frametitle{\titleTOC}
%	\tableofcontents
%\end{frame}
}

\section{Introduction}
\subsection{What is this Virtual Assistant?}

\begin{frame}\frametitle{The idea}
	Insert image with a diagram of the users, their roles and how the application connects them
\end{frame}

%\begin{frame}\frametitle{Section 1 - Subsection 1 - Page 2}
%	\begin{definition}
%		Let $n$ be a discrete variable, i.e. $n\in\Zset$.
%		A 1-dimensional periodic number is a function that depends periodically on $n$.
%		$$
%		u(n)=
%		[u_0,u_1,\ldots,u_{d-1}]_n=
%		\begin{cases}
%			u_0 & \mbox{ if $n\equiv 0 \pmod d$} \\
%			u_1 & \mbox{ if $n\equiv 1 \pmod d$} \\
%			\vdots \\
%			u_{d-1} & \mbox{ if $n\equiv d-1 \pmod d$}
%		\end{cases}
%		$$
%		$d$ is called the period.
%	\end{definition}
%\end{frame}

%\begin{frame}\frametitle{Section 1 - Subsection 1 - Page 3}
%	%this is too big.
%	\begin{example}
%		\centering
%		{
%		$$
%		\begin{array}{rcl}
%			f(n)
%			&=&
%			-\left[\frac{1}{2},\frac{1}{3}\right]_n n^2
%			+3n-[1,2]_n
%			\\
%			&=&
%			\begin{cases}
%				-\frac{1}{3} n^2 +3n-2
%				& \text{ if $n\equiv 0 \pmod 2$} \\
%				-\frac{1}{2} n^2 +3n-1
%				& \text{ if $n\equiv 1 \pmod 2$}
%			\end{cases}
%			% &=&
%			% -\frac{n^2}{2}+3n-1
%			% -\left\{ \frac{n}{2} \right\}
%			% \left( \frac{2}{3}n^2+2
%			% \right)
%		\end{array}
%		$$
%		\begin{minipage}{0.4\textwidth}
%			% insert picture (pdf file)
%			\includegraphics[width=\textwidth]{images/ex2_quasi_polynomial.pdf}
%		\end{minipage}
%		}
%	\end{example}
%\end{frame}

%\begin{frame}\frametitle{Section 1 - Subsection 1 - Page 4}
%	% Show first part of the screen highlighted
%	\begin{definition}
%		A polynomial in a variable $x$ is a linear combination of powers of $x$:
%		$$
%		f(x)=\sum_{i=0}^g c_i x^i
%		$$
%	\end{definition}
%	\pause
%	
%	% Show second part of the screen highlighted
%	\begin{definition}
%		A quasi-polynomial in a variable $x$ is a polynomial expression with periodic numbers as coefficients:
%		$$
%		f(n)=\sum_{i=0}^g u_i(n) n^i
%		$$
%		with $u_i(n)$ periodic numbers.
%	\end{definition}
%\end{frame}

\subsection{Section 1 - Subsection 2}

%\begin{frame}\frametitle{Section 1 - Subsection 2 - Page 1}
%	\begin{example}
%		\begin{columns}
%			\column{0.40\textwidth}
%			\centering
%			\includegraphics<1>[width=\textwidth]{images/ex3a_pp.pdf}
%			\includegraphics<2>[width=\textwidth]{images/ex3b_pp.pdf}
%			\includegraphics<3>[width=\textwidth]{images/ex3c_pp.pdf}
%			\includegraphics<4->[width=\textwidth]{images/ex3d_pp.pdf}
%			{ \textbf{\small{{$x+y\le p$}}}}
%			\column{0.1\textwidth}
%			
%			\begin{tabular}{c c}
%				$p$ & $f(p)$ \\ \hline
%				3 & 5 \\
%				\pause
%				4 & 8 \\
%				\pause
%				5 & 10 \\
%				\pause
%				6 & 13 \\
%			\end{tabular}
%			
%			\column{0.3\textwidth}
%			\pause
%			$$
%			\frac{5}{2}p+\left[-2,\frac{-5}{2} \right]_p
%			$$
%		\end{columns}
%	\end{example}
%\end{frame}

%\begin{frame}\frametitle{Section 1 - Subsection 2 - Page 2}
%	\begin{itemize}
%		\item <1-> The number of integer points in a \alert{parametric polytope} $P_{{p}}$ of dimension $n$ is expressed as a piecewise a quasi-polynomial of degree $n$ in ${p}$ (Clauss and Loechner).
%		
%		\item <2->
%		More general \alert{polyhedral counting problems}:\\
%		Systems of linear inequalities combined with $\lor, \land, \neg, \forall,$ or $\exists$ (Presburger formulas).
%		\item <3->
%		Many problems in \alert{static program analysis} can be expressed as polyhedral counting problems.
%	\end{itemize}
%\end{frame}

\subsection{Section 1 - Subsection 3}

%\begin{frame}\frametitle{Section 1 - Subsection 3 - Page 1}
%	% Just an example
%	A picture made with the package TiKz\\
%	\begin{example}
%		\centering
%		%Number of live elements = quasi-polynomial\\
%		\includegraphics[width=5cm]{images/abadab-anti-theta-01.pdf}
%		%$\Downarrow$ \\
%		%Memory usage = maximum over all execution points
%	\end{example}
%\end{frame}

\section{Second Section}

\subsection{Section 2 - Subsection 1}

%\begin{frame}\frametitle{Section 2 - subsection 1 - page 1}
%	\begin{alertblock}{Alertblock}
%		This page gives an example with numbered bullets (enumerate)\\
%		in an "Example" window:\\
%	\end{alertblock}
%	
%	\begin{example}
%		Discrete domain $\Rightarrow$ evaluate in each point\\
%		Not possible for\\
%		\begin{enumerate}
%			\item <1-> parametric domains
%			\item <2-> large domains (NP-complete)
%		\end{enumerate}
%	\end{example}
%\end{frame}

\subsection[]{Section 2 - Last Subsection}

%\begin{frame}\frametitle{Last Page}
%	\begin{block}{Summary}
%		\centering{End of the beamer demo\\
%		with a \emph{tidy} TU~Delft lay-out.\\
%		Thank you!}
%	\end{block}
%\end{frame}

\end{document}
