\chapter{Conclusion}

During the past fourteen weeks the AssisTU application has grown from a rough idea into a basic implementation that can be used to build upon in the 
future. It started off with an intensive research phase, where all possible angles, frameworks and implementation strategies were compared with each
other in order to be able to set out the most viable planning. Keeping in mind all the required features, the Model-View-Controller architecture was 
adopted in combination with the Java/Scala Play Framework.
The framework was used to design models which would cater to the application's requirements. The models predominantly included a User model, a Project 
model, a Calendar model, a Discussion model and a Task model. In this version the suggestion feature did not require any back-end functionality so a 
model for this feature was omitted. Roughly stated, each feature of this application was assigned a dedicated view, controller and model to provide low
coupling and high cohesion. Also for future reference and developers this seemed as the best way to go. \\

The goal of the AssisTU application was to provide students, researchers and teachers with an online environment that would help and guide them through
the process of writing academic documents by allowing them to interact with each other and exchange documents and feedback. For this purpose, a Project
feature was designed that would allow students to gain control over their writing process, and reviewers over their reviewing process. A project can be
seen as an online repository of documents with added functionality for the contributing users. These projects are closely related to a Calendar,
where students have the option to plan out their writing process. They can choose a template which comes with a predefined planning, or do their
own thing and plan their own planning. For the purpose of giving and receiving feedback, each project is assigned a discussion page where contributors
can easily discuss issues with each other in real-time. Reviewers also have access to the discussion page and can also give their feedback on topics
started by either students or themselves. A Task feature was added to the application as a user-specific `to-do' list. These tasks do not show up
on the Calendar, however the users do get notified by email about approaching deadlines. The final feature implemented is the suggestion page. This page,
for now, is mainly used for the administrators of the application to track and gather information about which topics students struggle with the most. In 
future releases however, this suggestions page will increase to become a more and more dynamic entity that will also be able to provide user/project 
specific tips and guidelines on how to write specific parts of an article. The vision for this is the use of Artificial Intelligence to detect writing 
bottlenecks and provide dynamic and automatic feedback.