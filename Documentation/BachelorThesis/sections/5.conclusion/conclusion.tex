\chapter{Conclusion}

During the past fourteen weeks the AssisTU application has grown from a rough idea into a basic implementation that can be used to build upon in the future. It started off with an intensive research phase, where all possible angles, frameworks and implementation strategies were compared with each other in order to be able to set out the most viable planning. Keeping in mind all the required features, the Model-View-Controller architecture was adopted in combination with the Java/Scala Play Framework.
The framework was used to design models which would cater to the application's requirements. The models predominantly included a User model, a Project model, a Calendar model, a Discussion model and a Task model. In this version the suggestion feature did not require any back-end functionality so a model for this feature was omitted. Roughly stated, each feature of this application was assigned a dedicated view, controller and model to provide low coupling and high cohesion. Also for future reference and developers this seemed as the best way to go. \\

The goal of the AssisTU application was to provide students, researchers and teachers with an online environment that would help and guide them through the process of writing academic documents by allowing them to interact with each other and exchange documents and feedback. For this purpose, a Project feature was designed that would allow students to gain control over their writing process, and reviewers over their reviewing process. A project can be seen as an online repository of documents with added functionality for the contributing users. These projects are closely related to a Calendar, where students have the option to plan out their writing process. They can choose a template which comes with a predefined planning, or do their own thing and plan their own planning. For the purpose of giving and receiving feedback, each project is assigned a discussion page where contributors can easily discuss issues with each other in real-time. Reviewers also have access to the discussion page and can also give their feedback on topics started by either students or themselves. A Task feature was added to the application as a user-specific `to-do' list. These tasks do not show up on the Calendar, however the users do get notified by email about approaching deadlines. The final feature implemented is the suggestion page. This page, for now, is mainly used for the administrators of the application to track and gather information about which topics students struggle with the most. In 
future releases however, this suggestions page will increase to become a more and more dynamic entity that will also be able to provide user/project specific tips and guidelines on how to write specific parts of an article. The vision for this is the use of Artificial Intelligence to detect writing bottlenecks and provide dynamic and automatic feedback. \\

To conclude this report, a comparison between the intended requirements and the final prototype will be discussed. For the functional requirements, nearly all requirements are fully met. Only two functional requirements turned out slightly different than what was proposed. The first one is the suggestion page that doesn't exactly give tips dynamically, but only statically. An agreement with the client was made to use the static tips and suggestions as an indicator of which type of information would be dynamically provided to the students in future releases. The second functional requirement that tranformed during the implementation is the logging. There is no automatic log file on which analytics can be done, however the client has full database access which can function as an alternative to a log file.
For the technical requirements, two requirements were omitted due to external factors. The first technical requirement that got omitted was the TU Delft NetID functionality. The University's IT-department was in the midst of transitioning their services to a more modern version, during which they were not able to provide us with the needed access information and clearance. A second technical requirement that is not met is the fully tested system. This is partially due to the previous unmet requirement. The communication with the IT department was troublesome as they were not straightforward about (eventually) not being able to register our application for their services. This resulted in one to two weeks lost as the supporting code was written awaiting the IT department's final reply. Another factor was that certain features were very hard to test in a testing environment such as Google Plus sign-in and document upload/download. Due to the combination of these factors with deadlines, an agreement with the coach was made to only partially test the application and invest the available time into the functional requirements.
Finally, all usability requirements were met for this application. \\

In conclusion, the AssisTU application became an excellent prototype implementation of the Virtual Assistant idea that was proposed to us by the TU Delft Library.