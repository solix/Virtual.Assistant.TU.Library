\chapter{Recommandation}
\section{TU Delft NetID}

\subsection{Authentication}

At the time this project was implemented, the NetID service of the TU Delft was undergoing an update and complete overhaul. 
Because of this transfer, the central IT department was unable to provide us with access information to use the NetID OAuth2 service.
Given that the application is primairily aimed at students and researches of the TU Delft, the implementation of the NetID
OAuth2 service should still remain a part of the future of this project. It will allow for a better integration with other
services provided by the TU Delft.

\section{Future Evolution}

\subsection{User Logging}

The vision for this application was not only to provide a supporting system for accessing and discussing files, but also create
a recommender-styled suggestion page that can identify the most common problems users face when writing papers and articles.
In order to create a good recommendation system, a minimum amount of data is needed to identify user interests and problems.
A good way to do this is to, for example, provide links to information from the suggestion page and then ask the user
how useful this information was for him or her on a scale of 1 to 5 or 1 to 10. Useful information can then be used to locate
or create even more useful links, whereas less useful information can be bumped from the recommendation system. Once the system
has gathered enough information, and can identify the most common struggles, you can teach it to give suggestions at the time
of relevance within the project. 