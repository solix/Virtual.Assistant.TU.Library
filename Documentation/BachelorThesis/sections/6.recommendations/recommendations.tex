\chapter{Recommendations}

\section{Authentication}

The following recommendations are all Authentication-oriented. Authentication covers all the possible ways a user can log
himself into the application with either a default account or a social account.

\subsection{TU Delft NetID}

At the time this project was implemented, the NetID service of the TU Delft was undergoing an update and complete overhaul. 
Because of this transfer, the central IT department was unable to provide us with access information to use the NetID OAuth2 service.
Given that the application is primarily aimed at students and researches of the TU Delft, the implementation of the NetID
OAuth2 service should still remain a part of the future of this project. It will allow for a better integration with other
services provided by the TU Delft.

\subsection{Mendeley}

The current implementation of the Mendeley document-sharing configuration is quite ubiquitous. A user now only has the option
to either share his complete library, or not at all. If the Mendeley library sharing functionality is kept in future releases,
a good recommendation would be to expand its functionality and allow users to have more control about which document gets shared
in which project. A possible implementation would be to have some Mendeley Control Center inside the application, where you
can specify which documents you wish to import, have an overview of which documents are shared in which projects, and maybe
even have the option to implicitly share documents between users directly.

\section{Functionality}

Functionality in this case means the way users have the ability to interact with the application.

\subsection{Asynchronousness}

The initial idea was to have asynchronous notifications, invitations, messages, etc., however this proved troublesome in certain
cases which resulted in not everything behaving in an asynchronous manner. For future implementations, it would be easier for users
if project invitations and notifications would appear asynchronous instead of after reloading. Also suggestions could be
triggered asynchronously in the future, for instance when the system notices the user might have some trouble somewhere with
a certain subject of his or her article, a small tip or suggestion could pop up from the bottom right.

\subsection{Calendar}

The current implementation of the calendar misses a bit of robustness and extended functionality, which leaves some room for
improvement. A first good recommendation would be to have the functionality to export the calendar to other applications such as
Google Calendar and iCal. A second one would be to improve the look and feel of the Calendar page, such as adding colourcodes to
indicate which event belongs to which project, or adding some gradients to the events so they are easier on the eyes.

\section{Data Gathering}

The following and final part of recommendations is about how the application could be able to learn from its user's behaviours,
and how administrators would be able to extract relevant information.

\subsection{User Logging}

The vision for this application was not only to provide a supporting system for accessing and discussing files, but also create
a recommender-styled suggestion page that can identify the most common problems users face when writing papers and articles.
In order to create a good recommendation system, a minimum amount of data is needed to identify user interests and problems.
A good way to do this is to, for example, provide links to information from the suggestion page and then ask the user
how useful this information was for him or her on a scale of 1 to 5 or 1 to 10. Useful information can then be used to locate
or create even more useful links, whereas less useful information can be bumped from the recommendation system. Once the system
has gathered enough information, and can identify the most common struggles, you can teach it to give suggestions at the time
of relevance within the project. 