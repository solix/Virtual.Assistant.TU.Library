The requirements provide the backbone to a project. They are confirmed in accordance with our client and our coach. It is imperative that these requirements are met at the end of the implementation.

\subsection{Functional Requirements} % (fold)
\label{sub:functional_requirement}

Below is a list of functional requirements that the application should meet:

\begin{enumerate}
	\item Multiple Users: Each project created by one user can be joined by multiple other users, where none or more joineing participants can be attributed owner privilege. The creator of the project will receive owner privilege by standard. The application should also facilitate reviews from supervisors or other members.
	\item Template: The application will provide several templates for the user, or the user can upload his of her own template
	\item Schedule: There has to be a module in which the users can schedule their writing progress according to their chosen template. (In terms of sections such as introduction etc.)
	\item Propose Suggestions/Tips: The system should be able to suggest tips and information to the user on how to write specific sections.
	\item Send/Receive Feedback: The application should facilitate feedback on a report written by a student from a reviewer.
	\item Done/Discard: For each suggestion or piece of feedback, the student can choose to check it off as `Done' or `Discarded'. When a user has checked `Done' or `Discarded' on the suggestion or piece of feedback, both the system and all other users should be notified. 
	\item Upload Document: Ability to upload the document into the application.
	\item Built-in Chat mechanism: To display the feedback from a reviewer and keep track of the conversations. 
	\item Versioning : Keep track of the versions of publications that a user has uploaded.
	\item Logging : Save the user's records in a separate log file for usage analytics.
\end{enumerate}

% subsection functional_requirement (end)
\subsection{Technical Requirements} % (fold)
\label{sub:technical_requirements}

Besides the functional requirements, which illustrate how the system should work in usage, there are also some technical requirements that illustrate how the system should be built. These technical requirements are listed below:

\begin{enumerate}
	\item Provide Mobile Support for Android/IOS
	\item Campus ID authentication: allow login with TUDelft netID
	\item Released as Open Source
 	\item System should be fully tested 
\end{enumerate}

% subsection technical_requirements (end)
\subsection{Usability Requirements} % (fold)

Finally there are also some usability requirements that should be met in order to guarantee a user-friendly application:

\label{sub:usability_requirements}
\begin{enumerate}
	\item The application must be fully functional on modern web browsers
	\item Efficient in use: System must facilitate efficiency of use for the user by providing information on the fly for the context
	\item Intuitiveness: User Interface must be intuitive and easy to use
\end{enumerate}
% subsection usability_requirements (end)