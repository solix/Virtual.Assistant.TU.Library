\section{SIG feedback} % (fold)
Throughout the implementation process of the application, the Software Improvement Group (SIG) has provided us with constructive feedbacks in terms of maintenance and quality, and also gave us some guidelines about how to improve the code in general. The application's code has been submitted to SIG at two stages during the development: Mid-review and final review.\\

By the end of the sixth sprint (the middle of the project), a first submission was performed after which SIG provided us with their initial feedback about the code. They stated that the code achieved a score of 4 out of 5 stars in their maintenance model, which is above average. The reason that the code did not achieve a full score was due component imbalance, duplication and unit size. As a side note, they also found the component structure in the file system unclear, which would make it difficult for other developers to gain a high level understanding of the application architecture.\\

For the second submission, the code was adapted according to SIG's feedback. The main goal for this submission was to abide to their guidelines, whilst continuously improving the application in terms of lines of code, files and components, which were growing simultaneously. Having applied SIG's feedback to the application has helped to improve maintainability, and improved the overall code quality.\\
 
In between these submissions, the unit size was kept as small as possible, mainly by avoiding code duplication. However it turned out this was not always possible. At the time of the final submission to SIG, the score did not improve despite the changes made compared to the previous submission. The main reason for this was limited time and sharp deadlines which shifted the focus from code cleanup and testing to implementing features that functions properly. Moreover, APIs where employed locally which were not self-written, yet still had to be included as a part of the file system for the application to function properly. On the bright side however, most of the duplicated code was removed, test coverage had been increased and the file structure was re-factored in order to provide a more clear overview of the MVC architecture.\\ 