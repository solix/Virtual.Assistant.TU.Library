\chapter{Interviews}    
 This appendix summarizes the interviews that were conducted from stakeholders within our project. The purpose of these was to gain insight into our problem domain and to find out what kind of features they would benefit from the most.

\section{Interview with a student}
Provided that the application itself is aimed at students, it was quite obvious that interviewing a student was necessary. For this interview, a student-colleague was approached and questioned about how he went about writing an article, what difficulties he encountered and what our application could provide that would make his writing process easier. One of the main issues he often encountered was that he often had trouble starting to write. He added that this was not because he could not think of anything, but because he could not easily narrow down how to write what he wanted to write. Especially at the start, wanting to write too much often resulted into a writers-block.
A second issue he had problems with was plagiarism. He claimed it's not always clear when something counts as plagiarism and when it doesn't. The academic world is not at all tolerant when it comes to (unintentional) plagiarism, and given the fact that it is often done because of incorrectly citing someone else's work, it becomes a bit of a barrier within a student's writing process.

\section{Interview with a reviewer}
At the heart of the application is the feature to guide students in their writing by providing them with tips, suggestions and feedback. These kinds of aides can be provided statically by referencing reading material, but also dynamically by involving a reviewer into the process who is able to provide very specific feedback. Therefore, an interview with a reviewer was conducted in order to find out how the application could facilitate their needs in terms of reviewing features. The reviewer told us that nearly all acts of reviewing in the academic world take the form of peer-reviews. During a peer review, the article gets distributed to other people from the same field after which they can assess it in terms of quality. Keeping this in mind, he suggested that the application might want to mimic this process, however tailored to the needs of students.


\section{Interview with Nicole Will - Head of Education Support}
 The most challenging part in this project is understanding how we can provide suggestions, tips, sources and other relevant information to students. In order to get a more detailed comprehension of this part within our problem domain, we conducted an interview with Nicole Will. Nicole works for the TU Delft Library as the Head of Education Support, and is also involved with the TULib website, which is an informative website about finding and using scientific information.
We wanted to know what the most valuable pieces of information were that we could provide students with when they write a report. She claimed there were three important parts that we should include, being the module on how to cite\cite{tulib:howtocite}, how to make a reference list, and how to use the APA\cite{tulib:apa} style for your references.
Other than these she also found it useful to provide the students with information on how to find information and articles, how to check for relevance and reliability of sources and what plagiarism is and how to avoid it.