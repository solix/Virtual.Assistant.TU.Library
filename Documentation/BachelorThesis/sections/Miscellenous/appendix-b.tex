\chapter{Interviews}    
 This appendix summarizes the interviews that were conducted from stakeholders within our project. The purpose of these was to gain insight into our problem domain and to find out what kind of features they would benefit from the most.

\section{Interview with **************}
Provided that the application itself is aimed at students, it was quite obvious that interviewing a student was necessary. For this interview, a student-colleague was approached and questioned about how he went about writing an article, what difficulties he encountered and what our application could provide that would make his writing process easier. An interview was conducted from **************, **************.
One of the main issues he often encountered was that he often had trouble starting to write. He added that this was not because he could not think of anything, but because he could not easily narrow down how to write what he wanted to write. Especially at the start, wanting to write too much often resulted into a writers-block.
A second issue he had problems with was plagiarism. He claimed it's not always clear when something counts as plagiarism and when it doesn't. The academic world is not at all tolerant when it comes to (unintentional) plagiarism, and given the fact that it is often done because of incorrectly citing someone else's work, it becomes a bit of a barrier within a student's writing process.
After explaining the purpose of the application, the student was inquired about what type of features he felt would be nice to have. The first few features he suggested all had to do with communication. He claimed it would be great to be able to communicate with each other on the same platform the documents would be shared. A (shared) planning would also be of help to him during his writing process. For the suggestion functionality, he claimed it would be nice only if it were handled correctly. There is an abundance of information available on-line on how to write articles, and finding the right information was usually a time-consuming effort. Therefor the suggestions would only be useful if they were concise and concrete with only a few great sources instead of many mediocre sources. 
In terms of usability, it should be able for people to connect to it in a quick and easy way. Using the NetID or some other common type of external authentication would make the application more attractive to use instead of having to sign up through a long and extended form. A final remark was that the fact that the application would be created by, or at least in close cooperation with, a university would benefit the application's credibility.

From this interview it became clear that students preferred the application to be easy to use and easy to access. Also the communication between collaborators should be a fundamental feature for the application, in terms of sharing, discussing and planning. 

\section{Interview with **************}
At the heart of the application is the feature to guide students in their writing by providing them with tips, suggestions and feedback. These kinds of aides can be provided statically by referencing reading material, but also dynamically by involving a reviewer into the process who is able to provide very specific feedback. Therefore, an interview with ************* was conducted in order to find out how the application could facilitate their needs in terms of reviewing features. The reviewer told us that nearly all acts of reviewing in the academic world take the form of peer-reviews. During a peer review, the article gets distributed to other people from the same field after which they can assess it in terms of quality. Keeping this in mind, he suggested that the application might want to mimic this process, however tailored to the needs of students.
After illustrating the function of the application, the reviewer was asked how it could facilitate the interaction between students and reviewers. The first fundamental issue would be organization. In many cases (when it comes to students), reviewers have more than one article to review, making it important to keep track of all communication file sharing between the students and the reviewers. If this application would at all be of any use to them, it would have to allow them to organize their supervised projects. A second issue would be communication. At this moment most communication between students and their supervisors happens through emailing back and forth. It would be useful for reviewers to be able to put their feedback in a place they know all the involved students can see it.

This interview again illustrated the need for straightforward communication between student and teacher. The application should also facilitate organizing the projects in a way that is easy for the reviewers to connect with the students.


\section{Interview with Nicole Will - Head of Education Support}
 The most challenging part in this project is understanding how we can provide suggestions, tips, sources and other relevant information to students. In order to get a more detailed comprehension of this part within our problem domain, we conducted an interview with Nicole Will. Nicole works for the TU Delft Library as the Head of Education Support, and is also involved with the TULib website, which is an informative website about finding and using scientific information.
We wanted to know what the most valuable pieces of information were that we could provide students with when they write a report. She claimed there were three important parts that we should include, being the module on how to cite\cite{tulib:howtocite}, how to make a reference list, and how to use the APA\cite{tulib:apa} style for your references.
Other than these she also found it useful to provide the students with information on how to find information and articles, how to check for relevance and reliability of sources and what plagiarism is and how to avoid it.