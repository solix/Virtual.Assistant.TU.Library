\subsection*{Users}

\paragraph{Authentication}

Users can authenticate themselves in two ways: a default account and Google Plus. A default account can be created from the signup page, where a 
user is asked for his first and last name, and his email address. After the user has submitted his information, a verification mail is sent to his email address. An unverified user has no access to the application, he will be redirected to the login page. The second type of authentication is through
Google Plus, where a user can identify himself through his Google account.

\paragraph{User roles}

The users of the AssisTU application can be divided into three roles: the owner (often the students), the reviewer and the guest. The owner role allows for the most control over a project. 
Owners can add or remove members, and (given the role of other participants) sometimes also files and discussions. Besides this, Owners are also allowed to edit the name, folder and description of a project of which they are the owner of, regardless of the initial creator. 
The reviewer role is reserved for teachers and supervisors. They get special attention within the application and can easily be visually identified by other users. Their contributions into a project in terms of files or comments cannot be deleted by anyone else but themselves, not even by the owners. 
The final role is the guest role. Guests are considered to be neither owners nor reviewers, the guest role is considered the role with the most restrictions.

\subsection*{Projects}

\paragraph{Projects}

The main component of the AssisTU application is the creation of, and participation in projects. A project is an entity in which owners, reviewers and guests come together. 
The project overview is represented as a series of panels with tables, separated by tabs at the top of the page. The name in the tab is the folder name, which is the same name users see when they join the project. 

\paragraph{Documents}

When a project page is rendered, the user gets to see the list of documents connected to this project. Every member of a project is able to download
any of these documents by clicking the download icon. Only owners and reviewers are able to upload their own documents.

\paragraph{Template}

A template is a combination of a file and series of calendarevents. When a user chooses to use a predefined template, he can download the templatefile
from the project page, and automatically all corresponding events will be loaded inside the calendar view. A user can also choose to create his own template. In this case the user can upload his own template file, and create his own events inside the calendar.

\subsection*{Calendar}

The calendar is where users can see the planning of all the projects of which they are an owner.
When a predefined template has been chosen, the events on the calendar will already be loaded after
a user has either created or joined a project as an owner. The calendar itself is personal, so users can also add their own events aside from their projectevents. When no predefined template was chosen, you can either
create your own template with personalised events, or have no template and do whatever you like.

\subsection*{Discussions}

The discussions page is a place where all members of a project are free to dicuss things with each other. It works the same way a chat would, where
each incoming message immediatly appears on another user's screen, only it looks a bit more advanced than just having messages pop up.
The discussion page shares the same top row of tabs as seen in the project view. Clicking one of these tabs
triggers the discussion overview to load for that specific project. In this overview, users gets a
list of subjects, along with the name of the original poster and an icon declaring his role within the project. 
By default all discussions are collapsed. Users can click the 'show/hide' button on a specific discussion to 
expand a discussion, which unveils its content and reactions (also called subcomments).

\paragraph{Leaving a comment or reaction}

The user can start a new discussion by clicking the 'start new discussion' button. A dropdown 
then appears with a visual template of how the message will appear in the list below after submission.  
Only when a discussion is expanded, an inputfield appears at the bottom which allows users to leave a reaction to this discussion. 
A reaction left on a message becomes instantly visible to other users who are browsing the same discussion without the need to reload the page to simulate direct conversation.

\paragraph{Attaching a file to a comment}

When a file gets uploaded to a project, a small textfield icon appears next to that users can click to start
a new discussion with that file attached. This brings them to a separate page where the same template appears
as from the 'start new discussion' dropdown. The only difference between this feature and the one above is that now an attachment appears at the end of the discussion. 
When submitted the user is taken to the discussion page where the attachment can be visually spotted underneath the subject from a collapsed message.

\subsection*{Tasks}

The tasks page is where the user can create his own personal 'todo' list. The view consists of two
panels next to each other where the left one contains tasks that the user still has to do,
and the right panel lists all the tasks that the user has already completed. A user can create
a new task by clicking on the green button with the plus-sign in the top left corner of the 'todo' task
panel. An input field appears inside the panel with a calendar icon. When the user has filled out
the task description, and has chosen a deadline on the calendar, he can click on the save button to have
it appear in the list of 'todo' tasks. This task object comes with an empty "check off" square on the left,
and the option to delete the task on the right. When the task is done, a user can cross it off his list by
clicking the check-off square in front of the task object. The task then appears crossed of on the panel on
the right where the user can choose to either redo the task, or remove it. Tasks do not appear in the calendar
view as per requirement by the client.

\subsecton*{Suggestions}

The suggestion page is not implemented in this version of the application, this version is meant to provide feedback to the creators
to determine how the suggestions should be built in the future. The main idea behind this page is to be able to guide users in their
process of writing a scientific paper or article. Suggestions should be both project-specific and user-specific, which was beyond
the scope of this project at this time.