\chapter{Introduction}

This document is the final report about the AssisTU application, a Bachelor Final Project performed by Arnaud Hambenne and Soheil Jahanshahi in
assignment of the TU Delft Library. 
The AssisTU application is created to help students write academic documents. When students and researchers want to write an article, a thesis or a 

dissertation, they may encounter many problems in terms of organizing, finding information on how to write their paper properly and have difficulties 
to plan the layout of their article. They may also encounter obstacles in terms of collaborating with others and bringing everyone in together one 
place. \\

Fundamentally, completing a research, article or dissertation paper takes effort, effective collaboration and proper planning. The lack of a supportive
online infrastructure is a problem that has been pointed out by the Research and Development department of TU Delft library. The R \& D department came 
up with some ideas about how to help researchers and students avoid major obstacles on writing their papers. The overall vision was to create some sort 
of Virtual Assistant in order to guide them throughout their process of writing an article.\\

The AssisTU web application is designed as a prototype that addresses this problem, and attempts to reflect the above mentioned vision in a more 
concrete way. It is designed as a `beta version' of the Virtual Assistant idea, meant to be built upon in the future.\\ 

In this document, a detailed summary about the `AssisTU' application is given. The following chapter documents the research phase that was conducted in 
order to determine the best suitable setup for the project, and which also explains the choice of tools that were selected to reach an optimal solution. 
Chapter three illustrates the design and structure of the application. The features of the application are explained in chapter four. The final chapters 
of this document summarize some facts, obstacles and recommendations followed by the conclusion.  