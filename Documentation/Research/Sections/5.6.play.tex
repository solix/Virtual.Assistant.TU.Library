One of the main reasons we selected the Play! Framework is because this framework has been proven in production. For example the LinkedIn web application has been developed using the Play! Framework. The following figure\ref{play} illustrates an overview of the Play! Framework.\\

\begin{figure}[h]
\centering
\includegraphics[scale=0.5]{./img/play.png}
\caption{\small{The main attributes and features of the Play! Framework}}
\label{play}
\end{figure}
 
 
The Play! Frameword offers a lot of documentation\cite{playDoc} that can help us to achieve our desired technical objectives. First of all, it is an open source application that has an amazing error handling because of the fact that everything is compiled and built on the fly. This make it easy for developers to detect bugs and can dramatically improve the developer's productivity to make a change. When reload your browser, you can immediately see the changes you just made, which makes it perfect for fast prototyping within our project. Moreover, Play! has great documentation available, it supports both the Scala and Java languages, it comes with a Scala template that allows you to write dynamic code within an HTML body, and it also comes with preconfigured testing support. This last one is particularly important because that means we do not need to worry about adding dependencies to the build file. \\
Play! also comes with the EBean server interface that makes it easy for fetching and saving beans to a particular DataSource. It can make reactive applications simpler because Play! is built on Netty, which means it supports non-blocking I/O. This will enable our web application to make remote calls inexpensively which is crucial for high performance web applications.






