\subsection{Domain Description} % (fold)
\label{sub:problem_description}

The TU Delft Library is an innovative library that is always looking for a creative solution for the future. The goal of this project is the technical implementation of a prototype of a "Virtual Assistant", which helps students and researchers to write a report, thesis or dissertation. This Virtual Assistant should provide its users with the ability to choose a layout template, allow for scheduling and feedback, and give suggestions or tips to users about how to write a report, thesis or dissertation.
This project should entail either native desktop software, a web application, or a mobile application. The final product should be released as open source software.

\subsection{Domain analysis} % (fold)

One of the key aspects that need to be well-researched for our project is our domain analysis. The domain analysis basically illustrates all the types of people involved with the problem domain (i.e. the stakeholders), and what their interests are. First we list all the stakeholders involved with our domain, secondly we define our scope, and finally we summarize a set of interviews we have conducted with people that embody relevant roles.

\subsubsection{Stakeholders} % (fold)
<<<<<<< HEAD

\paragraph{Client: TU Delft Library} The TU Delft Library is the entity that requested this project, and is therefor our client. They expect a prototype of a product that meets their requirements.

\paragraph{Users: Students and Researchers} The use of this product is aimed at students and researchers, which makes them our users. It is in our best interest to make the product as user-friendly as possible.

\paragraph{Contributors: Open Source Developers} Our client requested that our final version be released as an open source project. This entails some design and implementation constraints that benefit future developers.

\subsubsection{Scope \& Objectives}

The scope of our domain is set for the use-case of students who are writing a Bachelor project final report. One obvious reason for this scope is that this project is also a Bachelor final project, which will also include the process of writing a final report. This means that we will have affinity with the scope, which should imply an advantage within our design. Another reason is that there are plainly too many different cases and subjects within our problem domain to serve all of them accordingly within our deadline.
Our objectives 

=======
\begin{itemize}
	\item who is the stakeholders of the project?
	\item what are their roles?
	\item what are their issues
\end{itemize}
\subsubsection{Scope \& Objectives}
\begin{itemize}
	\item on line environment for supporting writing reports/dissertation
	\item owner of the project wanna have full access to information(suggestions,tips,sources,etc) related to the context of the paper
	\item stakeholders ability to interact seamlessly with each other
	\item Receive reviews from advisors/peers online.
\end{itemize}
>>>>>>> 30d696f48863f20f4c395d214c5d889031c6d8e3
\subsubsection{Interviews} % (fold)
\begin{itemize}
	\item Nicole Wills
	\item Some Reviewer
	\item owner of report(project)
\end{itemize}

In order to get a more detailed comprehension of our problem domain, we conducted several
interviews with people that represent relevant roles within our project. The first interview
is taken from Nicole Will, who works for the TU Delft Library and who's main function is 
providing students with information on how to write articles, reports or dissertations, and where to find relevant information. The second interview is taken from ************, who is a PhD Candidate at the TU Delft.

\paragraph{Nicole Will} ~



\paragraph{***********}

 